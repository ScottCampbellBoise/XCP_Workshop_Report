\chapter{Introduction}

\section{Philosophy of the Workshop}

The two primary purposes of LANL's Computational Physics Student
Summer Workshop are (1) To educate graduate and exceptional
undergraduate students in the challenges and applications of
computational physics of interest to LANL, and (2) Entice their
interest toward those challenges. Computational physics is emerging as
a discipline in its own right, combining expertise in mathematics,
physics, and computer science. The mathematical aspects focus on
numerical methods for solving equations on the computer as well as
developing test problems with analytical solutions.  The physics
aspects are very broad, ranging from low-temperature material modeling
to extremely high temperature plasma physics, radiation transport and
neutron transport. The computer science issues are concerned with
matching numerical algorithms to emerging architectures and
maintaining the quality of extremely large codes built to perform
multi-physics calculations. Although graduate programs associated with
computational physics are emerging, it is apparent that the pool of
U.S. citizens in this multi-disciplinary field is relatively small and
is typically not focused on the aspects that are of primary interest
to LANL. Furthermore, more structured foundations for LANL interaction
with universities in computational physics is needed; historically
interactions rely heavily on individuals' personalities and personal
contacts. Thus a tertiary purpose of the Summer Workshop is to build
an educational network of LANL researchers, university professors, and
emerging students to advance the field and LANL's involvement in it.

As before, the
workshop's goals were achieved by attracting a select group of
students recruited from across the U.S. and immersing them for ten
weeks in lectures and interesting research projects.  The lectures
provided an overview of the computational physics topics of interest
along with some detailed instruction while the projects gave the
students a positive experience accomplishing technical goals. Each
team consisted of two students working under one or more LANL mentors
on specific research projects associated with predefined
topics. The students'
growth was furthered by their participation on teams where their
teammates were sometimes of a different academic year.  It also
developed their skills by requiring them to produce written and oral
reports that they presented to peers, mentors, and management.

\chapter{Funding and Participation Profile}

\section{LANL Staff}

The Advanced Scientific Computing (ASC) Program at Los Alamos National
Laboratory sponsors this Summer Workshop by funding the workshop
coordinator and paying the lease for the workshop facility.  Funding
for the students' stipends come from a variety of programmatic
sources.  A large majority of them fall under various projects that
are part of the ASC Program, but a few other programs also provide
funding for some students.  This year, there were sixteen mentors
supervising twelve teams, which is currently the maximum allowed in
this workshop.  Mentors from XCP, CCS, and T participated\todo{Update
  divisions.}.  Broad participation is welcomed and it is hoped that
it continues in future years.


\section{Students}

\todo{Update the number of students}X students applied for
admission to the workshop, all eligible U.S. citizens with the
breakdown shown in the chart on the next page.  The twenty-two who
ultimately were selected and participated were from the following
schools: \todo{Add list of schools}

\begin{figure}
\centering
\missingfigure{Acceptance statistics}
% \includegraphics[scale=.90]{Introduction/graphics/ApplicantAnalysis_BarChart.png}
\caption{This figure shows the number of students who applied to the
  Summer Workshop and how many were accepted and participated, broken
  down by academic year.  In this figure ``G1'' means ``first-year
  graduate student'' at the time of the workshop,.i.e., starting their
  first year of graduate school in the fall after the workshop. ``G2''
  means ``second-year graduate student'' at the time of the workshop,
  i.e., starting their second year of graduate school in the fall
  after the workshop.}
\end{figure}

\subsection{Lectures}

In this sixth\todo{Update year.} year of the Summer Workshop, efforts
toward more tightly integrating the lecture series were continued.
The increased integration is part of an effort to transform the
stand-alone lectures into a sequence exhibiting a more course-like
feel.  A foundational lecture at the beginning of the Summer Workshop,
introducing the fundamentals of transport theory, was continued this
year to provide a common basis upon which several other lectures could
build.  Also the development of a one-dimensional hydrocode was
performed in class; the resulting code provided a basis for other
lecture materials and for exploratory studies that some of the
students performed at the beginning stages of their projects.  The
approximately 28 hours of lectures, for which the students' attendance
was required, were augmented with other lectures and demonstrations
for which the students' attendance was optional.  These lectures were
provided to help students who were lacking certain skills develop them
quickly to aide them during the summer.  The lectures included a
tutorial on C++ object-oriented programming, Python programming, and
Unix.

The lectures were scheduled to be most frequent in the beginning of
the workshop, when the students' research was just getting started and
they needed the most background information. Their frequency dropped
significantly until there were no lectures at all in the latter weeks
of the workshop so that the students could focus on their research.
The lectures are summarized in the table that follows.

\todo{Update table}
\begin{small}
\begin{center}
  \begin{tabular}{ |  l | c | l |  }
    \hline
\multicolumn{3}{|c|}{\bf{Required Lectures}} \\ \hline
\bf{Title}                                                                 &\bf{Hrs.}          &  \bf{Lecturer}                  \\ \hline \hline
Essentials of Transport Equations                                          &    2              &  S. Runnels                     \\ 
Introduction to Lagrange Hydro                                             &    1              &  S. Runnels                     \\ 
Intro to High-Performance Computing at LANL                                &    1              &  R. Cunningham                  \\ 
Introduction to Hydro Terminology and Artificial Viscosity                 &    1              &  S. Runnels                     \\ 
Survey of ALE Methods                                                      &    1              &  N. Morgan                      \\ 
Interface Reconstruction Methods                                           &    1              &  M. Shashkov                    \\ 
Introduction to Slidelines                                                 &    1              &  N. Morgan                      \\ 
Plasticity Modeling                                                        &    1              &  S. Runnels                     \\ 
Warm Dense Matter Simulation                                               &    1              &  O. Certik                      \\ 
Live Demo: Development of a 1-D Gas Hydrocode                              &    1              &  S. Runnels                     \\ 
Live Demo: Adding Plasticity to a 1-D Hydrocode                            &    1              &  S. Runnels                     \\ 
Introduction to Thermal Radiation Transport                                &    1              &  T. Urbatsch                    \\ 
Introduction to Molecular Dynamics                                         &    1              &  C. Starrett                    \\ 
Radiation Hydrodynamics                                                    &    1              &  S. Ramsey                      \\ 
Turbulence Modeling                                                        &    2              &  D. Israel                      \\ 
Opacity                                                                    &    1              &  C. Fontes                      \\ 
Sn Discretization Methods                                                  &    1              &  J. Hill                        \\ 
Galerkin Finite Element Method                                             &    1              &  S. Runnels                     \\ 
Introduction to Monte Carlo and MCNP                                       &    3              &  F. Brown                       \\ 
Mimetic Methods for Diffusion                                              &    1              &  S. Runnels                     \\ 
V \& V and Uncertainty Quantification                                      &    1              &  G. Weirs (Sandia)              \\ 
Hypervelocity Impact Short Course Highlights                               &    3              &  J. Walker (SwRI)               \\ \hline
\multicolumn{3}{|c|}{\em{Optional Lectures}} \\ \hline
Live Demo: Tutorial in C++ Programming                                     &    2              &  S. Runnels                     \\ 
Live Demo: Unix Tutorial                                                   &    1              &  S. Runnels                     \\ 
Introduction to Python                                                     &    1              &  D. Israel                      \\ 
Python, Git, and Jupyter Notebooks                                         &    1              &  O. Certik                      \\  \hline
  \end{tabular}
\end{center}
\end{small}

%\bibliographystyle{plainnat}
%\bibliography{references}
