\documentclass[]{article}

\usepackage[margin=1in]{geometry}
\usepackage{amsmath}  % needed for \tfrac, \bmatrix, etc.
\usepackage{amsfonts} % needed for bold Greek, Fraktur, and blackboard bold
\usepackage{graphicx} % needed for figures
\usepackage{subcaption}

\begin{document}
	\title{XCP Computational Physics Workshop Final Report Draft}
	\author{Scott Campbell}
	\date{\today}
	
	\maketitle
	
\section{Chapter Abstract}
	The response function based variance reduction technique outlined is for use in Implicit Monte Carlo (IMC) Thermal Radiation Transport (TRT) problems. It uses first a backwards approximation to fuel a forward approximation to improve the convergence, reliability, and accuracy of the IMC simulations. We were able to demonstrate the improvements resulting from the integration of this method into the open-source branson IMC code developed by Alex Long (\textit{along@lanl.gov}) using modern object-oriented languages. 
	
	This report outlines the background, motivation and theory for the method, the algorithmic procedure, initial results, and an analysis of the cases and conditions where the method is useful. 
	
\section{Introduction and Theory}
	\subsection{Thermal Radiation Transport}
		The scattering and absorbtion of photons emitted from a material are described by the thermal radiative transfer (TRT) equations: 
		\begin{equation} \label{Eq: TRT_1}
		\frac{1}{c} \frac{\partial I}{\partial t}(\vec{r}, \vec{\Omega}, \nu, t) + \vec{\Omega} \frac{\partial I}{\partial \vec{r}}(\vec{r}, \vec{\Omega}, \nu, t) + \sigma_{a}(\vec{r}, \nu, T)I(\vec{r}, \vec{\Omega}, \nu, t) = 2 \pi \sigma_{a}(\vec{r}, \nu, T)B(\nu, T) + \frac{Q}{2}(\vec{r}, \nu, t),
		\end{equation}
		\begin{equation} \label{Eq: TRT_2}
		c_{v}(\vec{r}, T) \frac{\partial T}{\partial t}(\vec{r},t) = \int_{0}^{\infty} \int_{-1}^{1} \sigma_{a}(\vec{r}, \nu^{\prime}, T)[I(\vec{r}, \vec{\Omega}^{\prime}, \nu^{\prime}, t) - 2 \pi B(\nu^{\prime}, T)] d \vec{\Omega}^{\prime} d \nu^{\prime}
		\end{equation}
		where $I$ is the specific intensity, $T$ is the material temperature (keV), $c$ is the speed of light, $B$ is Plank's radiation function, $Q$ is the inhomogenous source, $c_{v}$ is the material's specific heat, and $\sigma_{a}$ is the absorption opacity. Each of the terms in Eq. \ref{Eq: TRT_1} corresponds to a loss or gain term. The first term describes how the specific intensity is dependent on different gains and losses. The second is the streaming term describing how photons are lost by streaming out of the phase space. The third describes the loss due to absorption into the material. On the right-hand side, the first is a gain term describing the radiation source from material temperature, and the second term then describes background radiation. Equations \ref{Eq: TRT_1} and \ref{Eq: TRT_2} are non-linearly coupled by the material temperature. 

	\subsection{Implicit Monte Carlo}
		Monte Carlo methods are used to model time-dependent, nonlinear, radiative transfer problems in complex three-dimensional configurations. This stochastic numerical method uses random sampling to determine where and how a particle moves through a material. Additionally, this method minimizes the effects of discretization errors through a continuous treatment of energy, space, and angle leaving the primary errors to be rooted in stochastic uncertainties. However, Monte Carlo comes at the expense of long run times (convergence of $\frac{1}{\sqrt{N}}$) and heavy use of machine respources. The problem domain space is discretized into rectangular regions refered to as cells. Particles then move through the cells and can be either absorbed into the material, scattered, or they can leave the problem domain. Additionally, a particle is transported until it reaches the end of the time step, or is fully absorbed into the material. Information about the system (e.g. fluence) are stored in a 'tally'.
				
		The Implicit Monte Carlo (IMC) method was developed by Fleck and Cummings in 1971. IMC uses `effective scattering' to model a particle's absorbtion/re-emission in a material for its current time step. This is represented by the fleck factor, $f$, described as follows:
		\begin{equation}
			f = \frac{1}{1 + \frac{4acT^{3}\sigma \Delta t}{c_{v}}}
		\end{equation}
		where $a$ is the radiation constant, $c$ is the speed of light, $T$ is the material temperature, $\sigma$ is the material opacity, $\Delta t$ is the time step, and $c_{v}$ is the material specific heat.
		
		This method applies two note-worthy approximations: semi-implicit discretization of time, and the linearlization of the thermal radiative transfer equations. By linearizing the thermal radiative transfer equations, IMC is known to lead to inaccurate or non-physical results. Additionally, the time discretization is not fully implicit as implied, but rather semi-implicit as it is typically too expensive to converge otherwise. 
			
	\subsection{Variance Reduction for IMC}
		Since Monte Carlo is a stochastic method, there will always be some uncertainty. Inherent drawbacks to IMC methods include slow convergence and large computational requirements [CITE]. Furthermore, IMC is known to break down for complex systems and geometries. Additionally, in cases where a material is optically thick and the distance from source to observation (i.e. a tally surface) is large enough, traditional IMC will fully absorb the particles into the material before they reach the tally, leaving the tally poorly sampled.
		
		Due to these issues, variance reduction methods for IMC methods are necessary to provide equivalent answers while using less resources and converging faster. In the IMC code used, three standard variance reduction techniques were incorporated in addition to the response function method. 
		
		\subsubsection{Implicit Capture}
			In implicit capture or absorbtion supression, particles are not allowed to be fully absorbed into the material. Instead, at each 'event' (e.g. scattering or boundary crossing) the weight of the particle is adjusted according to the following relationship:
			\begin{equation}
				w_{n,~i+1} = w_{n,~i} (1 - \frac{\sigma_{a}}{\sigma_{a} + \sigma_{s}})
			\end{equation}
			where $\sigma_{a}$, $\sigma_{s}$ are the absorbtion and scattering cross-sections respectively. Using the fleck factor, the above becomes
			\begin{equation} \label{Eq: new_E}
				E_{particle,~i+1} = E_{particle,~i} e^{-\sigma_{a} \cdot f \cdot d_{event}} .
			\end{equation}
			where $d_{event}$ is the distance to the next position the particle travels to. Absorbtion suppression allows particles to stay active indefinately, greatly increasing the chance that the particle will make it to the tally surface. However, this method adds significant time complexity as particles then must be continually tracked. To counteract this, a history termination method is used to stop tracking the particle and reduce the complexity. Absorbtion suppression doesn't always guarantee that a tally surface will be well sampled in the case of a high scattering opacity that can mitigate the escaping flux.
			
			History termination methods `kill' particles once its energy falls below a set threshold. This method does increase variance, but more significantly reduces the computational complexity by not tracking particles once they no longer significantly contribute to the tally surface.
		
		\subsubsection{Source Biasing}
			Source biasing alters the distribution of the source particles, such that more particles are produced in regions of the space-angle-energy space that are of interest and importance to a particlular problems result. Replacing an unbiased source distribution, $S(\vec{r}, \vec{\Omega}, E)$, with the biased distribution, $S^{\prime}(\vec{r}, \vec{\Omega}, E)$. To produce a particle with a weight of $1$ while maintaining unbiased estimators, the initial particle weight would be 
			\begin{equation}
				w = \frac{S(\vec{r}, \vec{\Omega}, E)}{S^{\prime}(\vec{r}, \vec{\Omega}, E)}
			\end{equation} 
			for some initial $\vec{r}$, $\vec{\Omega}$, and $E$. This method is useful primarily for problems that have a strong dependence on particle energy and directionality, however, this is not representative of the large marjority of problems and neglects to address the complete absorption of particles that leads to a poory sampled tally.
		
	\subsection{Response Function Theory}
		A response function is thought to be an effective variance reduction method (VRM) since it contributes an adjusted energy value to the tally at every scatter event, rather than adding a contribution only once the particle passes through the tally surface. Because of this feature, the tally surface should be well sampled using the response function compared to standard variance redution methods. 
		
		A response function attempts to calculate the probability that a particle `survives' to the tally surface following its current trajectory. The weight of the particle based on this probability and current energy should the particle reach the tally surface is calculated by
		
		The response function is generated by tracing a number of particles through the problem domain starting uniformly on the tally surface, directed at a cosine-distribution angle. Since each cell may have a unique $\sigma_{a}$, every possible path from the source to the tally surface will result in a potentially unique contribution to the tally. The response function attempts to model this by averaging the absorption opacity based on the accumulation of all weighted $\sigma_{a}$ values from the previous cells into an effective opacity, $\sigma_{r}$. At every scattering event, the particles contribution to the tally is calculated by
		\begin{equation}\label{Eq: tally_contr}
		E_{contribution} = E_{particle}e^{-(\sigma_{r} + \frac{1}{c \Delta t})d_{tally}}
		\end{equation}
		where $E_{particle}$ is the current energy of the particle, $\Delta t$ is the current time step, and $d_{tally}$ is the distance of the particle to the tally surface along its path. 
		
\section{Method and Technical Approach}
	The implementation of the response function variance reduction method largely follows a standard Monte Carlo approach. At the start of the problem, several parameters are defined base on the properties of the materials in the problem domain. The absorption opacity, $\sigma_{a}$, is calculated using the fleck factor:
	\begin{equation}
	\sigma_{a} = f \sigma.
	\end{equation}The source emission, $S_{e}$ for each cell is defined as 
	\begin{equation}
		S_{e} = c \cdot a \cdot \Delta t \cdot \sigma_{a}  \cdot T^{4},
	\end{equation}
	and total source emission, $S_{e,~total}$, is the sum of $S_{e}$ for each cell;
	\begin{equation}
		S_{e,~total} = \sum_{i = 1}^{n_{cell}} S_{e,~n}.
	\end{equation}
	The normalized weight, $w_{ideal}$, of each particle in the simulation is the quotient of $S_{e,~total}$ and $N_{particles}$. The number of particles to be emitted in each cell is then the quotient of $S_{e}$ and $w_{ideal}$. A check is run to ensure that each cell emits at least one particle. The time of emission for each particle is determined from a uniform distribution over the time step:
	\begin{equation}
		t_{0,~particle} = \xi \Delta t
	\end{equation}
	where $\xi \in [0,1]$ is a uniformly distributed random number. The temperature of each cell is calculated by
	\begin{equation} \label{Eq: cell_T}
		T_{cell} = \rho c_{v} \Delta t \Delta E
	\end{equation}
	where $\rho$ is the material density and $\Delta E$ is the differnece between $S_{e}$ and the absorbed energy in each cell.

    At the beggining of each time step, the response function is generated for the mesh. Our approach discretizes the function over the domain space, using each cell as a region to calculate the response opacity, $\sigma_{r}$. To calculate $\sigma_{r}$ for each cell, a set number or particles, $N_{response}$, are traced over the mesh accumulating information. Each particle is initialized to start on the tally surface, and directed towards the problem source using a cosine-distributed angle. For each particle, the following information is recorded as the particle passes a distance $d_{cell}$ through each cell: 
	\begin{enumerate}
		\item The total distance the particle has traveled through each cell, $d_{total,~particle} = \sum d_{cell}$.
		\item The sum of the distance the particle travels through each cell multiplied by the cells absorbtion opacity, $\sigma d_{total,~particle} = \sum d_{cell} \sigma_{a,~cell}$.
		\item The total distance, $d_{total,~cell} = \sum d_{n,~cell}$ for every particle $n$ that passes through each cell.
		\item The total distance multiplied by the absorbtion opacity, $\sigma d_{total,~cell} = \frac{\sigma d_{total,~particle}}{d_{total,~particle}} d_{cell}$ for every particle that passes through each cell.
	\end{enumerate}
	where $\sigma_{a,~cell}$ is the absorbtion opacity of the cell the particle is in. The response value for each cell is then calculated by the following equation:
	\begin{equation}
		\sigma_{r} = \frac{\sigma d_{total,~cell}}{d_{total,~cell}}.
	\end{equation}
	
    Every particle sourced in a cell over a time step is then transported through the problem using the following scheme:  
	\begin{enumerate}
		\item Upon its creation, a contribution to the tally is calculated using Eq. \ref{Eq: tally_contr}.
		\item For each particle, while the particle remains in the problem domain, calculate:
			\subitem The distance to the next scattering event, $d_{scatter} = -\log{\frac{\xi}{(1 - f)(\sigma_{a} + \sigma_{s})}}$ where $\xi \in [0,1]$ is a uniformly distributed random number, distance to the cell boundary, $d_{boundary}$, distance to the tally surface, $d_{tally}$, and the distance to reach the end of the timestep, $d_{census}$.
			\subitem The distance to the next `event,' $d_{event} = min(d_{scatter}, d_{boundary}, d_{census})$.
			\subitem Move the particle a distance $d_{event}$, and reduce the particle weight according to Eq. \ref{Eq: new_E}.
	     	\subitem If the particle energy is below a threshold, stop tracking it.
			\subitem If $d_{event} = d_{scatter}$, the new direction is sampled from a cosine distribution and a contribution to the tally is found from Eq. \ref{Eq: tally_contr}.
			\subitem If $d_{event} = d_{boundary}$, update the cell information to be the cell the particle is moving into.
			\subitem If $d_{event} = d_{census}$, push the particle into a list containing all particles that `survived' the time step and stop tracking the particle.
		\item At the end of the time step after all particles have been transported, calculate the energy of all the census particles and redistribute it randomly to a smaller number of representative particles.
		\item Additionally, the temperature $T$ of each cell is updated at the end of each timestep according to Eq. \ref{Eq: cell_T}.
			
		\item Update the time step information, and repeat the above scheme for all time steps.
	\end{enumerate}

\section{Verification}


\section{Results}


\section{Conclusions}


\end{document}
