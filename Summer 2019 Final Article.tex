\documentclass[]{article}

\usepackage[margin=1in]{geometry}
\usepackage{amsmath}  % needed for \tfrac, \bmatrix, etc.
\usepackage{amsfonts} % needed for bold Greek, Fraktur, and blackboard bold
\usepackage{graphicx} % needed for figures
\usepackage{subcaption}


\chapter{Variance Reduction at Scale for Implicit Monte Carlo Thermal Radiative Transfer}

\AuthorBlock{Scott E. Campbell}{Mathew A. Cleveland, Kendra P. Long, and Ryan T. Wollaeger}

\begin{chapabtract}
This project utilized the Branson Monte Carlo mini-app \ref{branson_git}
\end{chapabstract}

\section{Introduction}
	\subsection{Implicit Monte Carlo}
		Monte Carlo methods are used to model time-dependent, nonlinear, radiative transfer problems in complex three-dimensional configurations. Additionally, they methods minimize the effects of discretization errors through a continuous treatment of energy, space, and angle leaving the primary errors to be rooted in stochastic uncertainties. The Implicit Monte Carlo (IMC) method applies to approximations: semi-implicit discretization of time, and the linearlization of the thermal radiative transfer equations. The general procedure for the IMC is outlined in \ref{}. ADD SECTION ABOUT THE ISSUES OF THE IMC APPROXIMATIONS
				
		For thermal transport problems, further variance is introduced based on the quantity of photons emitted in the system as well as the material opacity. In cases where a material is optically thick and the distance from source to observation (i.e. a tally surface) is large enough, traditional IMC will fully absorb the particles into the material before they reach the tally. Consequently, the number of particles that contribute to the tally is significantly less than the number that is emitted, thus increasing the variance of the approximation.  
	
	\subsection{Standard Variance Reduction Techniques}
		Absorbtion supression is one of the most common variance reduction methods. In essence, particles are never absorbed into the material: at each event, the weight of the particle is adjusted according to
		\begin{equation}
			w_{n, i+1} = w_{n, i} * (1 - \frac{\sigma_{a}}{sigma_{a} + sigma_{s}})
		\end{equation}
		where $\sigma_{a}$, $\sigma_{s}$ are the absorbtion and scattering cross-sections respectively. Absorbtion suppression allows particles to stay active longer, increasing the chance that the particle will make it to the tally surface. To be useful, this method is often used in combination with a history termination method.
		
		History termination simply `kills' particles once its energy falls below a set threshold. This method helps reduce the computational complexity by not tracking particles once they no longer significantly contribute to the tally surface. The russian roulette method is a common way to implement history termination since it is an unbiased method. ADD REFERENCE HERE
			
\section{Variance Reduction via Response Functions}
	A response function reduces the variance of the approximation by effectively increasing the number of particles that contribute to the tally. A response function is generated at the beginning of each time step in the simulation by tracing rays from the tally surface towards the source. As the photon moves through the discretized spatial cells, four pieces of information are recorded:
	\begin{enumerate}
		\item The total distance the particle has traveled, $distance_{total, photon} = \sum_{i = 1}^{n} d_{cell}$.
		\item The sum of the distance the particle travels through each cell multiplied by the cells absorbtion opacity, $\sigma distace_{total, photon} = \sum_{i = 1}^{n} d_{cell}*\sigma_{a, cell}$.
		\item The total distance of all particles that pass through each cell, $distance_{total, cell} = \sum_{i = 1}^{m} d_{m, cell}$.
		\item The total distance of all particle passes through each cell multiplied by the adjusted photon absorbtion opacity, $\sigma distance_{total, cell} = \sum_{i = 1}^{m} \frac{\sum_{i = 1}^{m} d_{m, cell}*\sigma_{a, cell}}{d_{m, cell}} d_{m, cell}$.
	\end{enumerate}
	Where $n$ is the number of cells, $m$ is the total number of particles being traced, $d_cell$ is the distance the photon travels through the cell, and $\sigma_{a, cell}$ is the absorbtion opacity of the cell. The response value for each cell, the effective absorbtion opacity that it would encounter to reach the tally surface, is then calculated by the following equation:
	\begin{equation}
		$\sigma_{response} = \frac{\sigma distance_{total, cell}}{distance_{total, cell}}$.
	\end{equation}
	At the creation of each particle, and at every scattering event the contribution of the particle to the tally is determined by
	\begin{equation}
		tally contribution = E_{particle} e^{-(\sigma_{response} + \frac{1}{c \Delta t}) d_{tally}}
	\end{equation}
	Where $E_{particle}$ is the particles energy, $c$ is the speed of light, $\Delta t$ is the time step, and $d_{tally}$ is the distance of the particle to the tally surface along its path. The advantage of this method is that a much larger number of particles contribute to the tally. 
	
\section{}












References:

Branson Git Repo: branson_git

	